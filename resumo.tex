%================================= Resumo e Abstract ========================================
\chapter*{Resumo}


\begin{quotation}
\noindent Embora a propagação de feixes ópticos seja um tema muito investigado, existe ainda uma grande variedade de estudos a se efetuar, principalmente no desenvolvimento de métodos que permitam, de forma analítica, a descrição exata da enorme diversidade de feixes com propriedades distintas.

A principal contribuição desta dissertação é a proposta de uma metodologia matemática para a obtenção de feixes escalares e eletromagnéticos não paraxiais puramente propagantes como soluções analíticas exatas da equação de onda e das equações de Maxwell. Tal método baseia-se em uma solução analítica para as integrais que descrevem superposições de feixes de Bessel de ordem zero (não evanescentes) com qualquer tipo de função espectral. Exemplos de feixes não paraxiais são apresentados para a validação do método proposto neste trabalho, os quais provam a grande eficiência em termos do pouco esforço computacional quando são comparados com os métodos de outros autores.

\vspace*{0.5cm}

\noindent Palavras-chave: Feixes ópticos não paraxiais. Aproximação paraxial. Feixes de Bessel. Transformada de Fourier. Equação de onda. Equações de Maxwell. Polarização linear, azimutal e radial.

\end{quotation}


\chapter*{Abstract}


\begin{quotation}


\noindent Although the propagation of optical beams has been vastly studied, there is still a huge amount of research topics to be exploited, mainly regarding the developing of exact analytic methods.

The main contribution of this work is the development of a mathematical methodology to obtain nonparaxial propagating scalar and electromagnetic beams as exact analytic solutions of the wave equation and Maxwell's equations, respectively. This method is based on an very general solution to the continuous superposition of zero order Bessel beams (non-evanescent) with any kind of spectral function. Examples of non paraxial beams are shown to validate the method proposed in this work, which proves to be very efficient, based on low computational effort when compared to other author's methods.

\vspace*{0.5cm}

\noindent Key-words: Nonparaxial optical beam. Paraxial approximation. Bessel beams. Fourier transform. Wave equation. Maxwell's equations. Linear, azimuthal and radial polarization.

\end{quotation}

\null

